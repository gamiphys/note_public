\documentclass[a4paper,11pt]{jsarticle}

% 数式
\usepackage{amsmath,amsfonts}
\usepackage{amsthm}
\usepackage{bm}
\usepackage{mathtools}
\usepackage{amssymb}

% 表
\usepackage[utf8]{inputenc}
\usepackage{diagbox} % 斜線付きセルを作成するために必要
\usepackage{booktabs} % 表の罫線を美しくするために必要
\usepackage{hhline} % 水平罫線を制御するために必要

% 画像
\usepackage[dvipdfmx]{graphicx}
\usepackage{ascmac}
\usepackage{physics}
\usepackage{float} % 追加

% 図
\usepackage[dvipdfmx]{graphicx}
\usepackage{tikz} %図を描く
\usetikzlibrary{positioning, intersections, calc, arrows.meta,math} %tikzのlibrary

% ハイパーリンク
\usepackage[dvipdfm,
  colorlinks=false,
  bookmarks=true,
  bookmarksnumbered=false,
  pdfborder={0 0 0},
  bookmarkstype=toc]{hyperref}

% 式番号を章ごとにリセット
\numberwithin{equation}{section}

\begin{document}

\title{B4ゼミ\#5}
\author{大上由人}
\date{\today}
\maketitle

\setcounter{section}{3}
\section{連続空間における確率過程}
前節では、離散状態に対する確率過程について述べたが、今回は連続時間の確率過程を考える。
\subsection{数学的基礎}
\subsubsection{Wiener過程}
標準的な連続空間のMarkov過程として、Wiener過程が知られている。
\begin{itembox}[l]{\textbf{Def.Wiener過程}}
    Wiener過程$\hat{W}(t)$は、
    \begin{align}
        P(\hat{W}(t+\Delta t)=x \mid \hat{W}(t)=x') = \frac{1}{\sqrt{2\pi \Delta t}} \exp(-\frac{(x-x')^2}{2\Delta t})\\
        P(x,0)=\delta(x)
    \end{align}
    を満たす確率過程である。
\end{itembox}
Wiener過程が確かに存在することを直感的に確かめる。一次元格子におけるランダムウォークを考える。格子定数を$a$、時間間隔を$\Delta \tau$とする。
各界のステップでは、確率$1/2$で右に$a$、$1/2$で左に$a$移動するものとする。すなわち、
\begin{align}
    T_{x \to x+a} = T_{x \to x-a} = \frac{1}{2}
\end{align}
とする。初期条件を$P(x,0)=\delta_{x,0}$とすると、時刻$t$における$x$の期待値は常に0である。
このとき、拡散定数は、
\begin{align}
    D &:= \frac{d\langle x^2 \rangle}{dt} \\
  &= \sum_{x} \frac{ x^2 P(x,t + \Delta t) - P(x,t)}{\Delta t} \\
  &= \sum_{x} \frac{x^2}{\Delta t} \left[ P_{x-a \rightarrow a} P(x - a, t) + P_{x + a \rightarrow a} P(x + a, t) - P(x,t) \right] \\
  &= \sum_{x} \frac{x^2}{\Delta t} \left[ \frac{1}{2} P(x - a, t) + \frac{1}{2} P(x + a, t) - P(x,t) \right] \\
  &= \frac{1}{\Delta t} \left[ \frac{1}{2} \sum_{x} x^2 P(x - a, t) + \frac{1}{2} \sum_{x} x^2 P(x + a, t) - \sum_{x} x^2 P(x,t) \right] \\
  &= \frac{1}{\Delta t} \left[ \frac{1}{2} \sum_{x'} (x' + a)^2 P(x', t) + \frac{1}{2} \sum_{x''} (x'' - a)^2 P(x'', t) - \sum_{x} x^2 P(x,t) \right] \\
  &= \frac{1}{\Delta t} \sum_{x} \left[ \frac{(x - a)^2 + (x + a)^2}{2} - x^2 \right] P(x,t) \\
  &= \frac{a^2}{\Delta t} 
\end{align}

このとき、確率分布は、中心極限定理により、ガウス分布に収束する。
\begin{itembox}[l]{\textbf{Recall.中心極限定理}}
    $N$個の独立同一分布の確率変数$X_i$の平均値は、$N \to \infty$で、平均$\mu$、分散$\sigma^2/N$の正規分布に収束する。
    ただし、$\mu$は$X_i$の平均、$\sigma^2$は$X_i$の分散である。

\end{itembox}
TODO:証明つける

Wiener過程の極限として白色Gaussノイズを導入する。
\begin{itembox}[l]{\textbf{Def.白色Gaussノイズ}}
    白色ガウスノイズは、以下のように与えられる。
    \begin{align}
        \hat{\xi}(t) = \lim_{\Delta t \to 0} \frac{\hat{W}(t+\Delta t) - \hat{W}(t)}{\Delta t}
    \end{align}
\end{itembox}
また、のちの便宜のため、時間$\Delta t$の間で離散化した白色ガウスノイズを
\begin{align}
    \hat{\xi}_{\Delta t}(t) \coloneq \hat{W}(t+\Delta t) - \hat{W}(t)
\end{align}
と定義する。
このとき、
\begin{align}
    \ev{\hat{\xi}(t)} &= 0 \\
    \ev{\hat{\xi}(t) \hat{\xi}(t')} &= \delta(t-t') 
\end{align}
が成り立つ。\\
$(\because)$\\
TODO:証明つける

また、Wiener過程を
\begin{align}
    \hat{W}(\tau) = \int_0^{\tau} \dd{t} \hat{\xi}(t)
\end{align}
と復元できる。上記の関係を、正式に、
\begin{align}
    \dd{\hat{W}(t)} = \hat{\xi}(t) \dd{t}
\end{align}
と表す。

特に、Wiener過程の二乗はアンサンブル平均をとることなく
\begin{align}
    (\dd{\hat{W}(t)})^2 = \dd{t}
\end{align}
が成り立つ。より正確には、$(\dd{\hat{W}(t)})^2$による積分は、二乗平均の極限の意味で普通の時間積分と同等である。
  \footnote{%
\(\lim_{\Delta t \to 0} g_{\Delta t} = g\) が平均二乗収束の意味で成り立つとは、  
\[
\lim_{\Delta t \to 0} \left\langle (g_{\Delta t} - g)^2 \right\rangle = 0
\]
が成り立つことをいう。
}
\begin{itembox}[l]{\textbf{Thm.Ito則}}
  \begin{align}
    \lim_{\Delta t \to 0} \left\langle
      \left( \sum_{n=0}^{N-1} \hat{\xi}_{\Delta t}(n\Delta t)^2 f(n\Delta t)
      - \int_0^{\tau} \dd{t} f(t) \right)^2
    \right\rangle &= 0, \\
    \lim_{\Delta t \to 0} \left\langle
      \left( \sum_{n=0}^{N-1} \hat{\xi}_{\Delta t}(n\Delta t) \Delta t f(n\Delta t)
      \right)^2
    \right\rangle &= 0, \\
    \lim_{\Delta t \to 0} \left\langle
      \left( \sum_{n=0}^{N-1} \hat{\xi}_{\Delta t}(n\Delta t)^k f(n\Delta t)
      \right)^2
    \right\rangle &= 0,
  \end{align}
      ここで \( N := \tau / \Delta t \), \( k \geq 3 \) とする。これらの関係式は形式的には次のように書ける:
      
      \begin{align}
        (\dd{\hat{W}(t)})^2 &= \dd{t}, \\
        \dd{\hat{W}(t)} \dd{t} = (\dd{\hat{W}(t)})^k &= 0.
      \end{align}
      
\end{itembox}
\textbf{Prf.}\\
$\xi_{\Delta t}$ の 4 次のモーメントは、
\begin{align}
\left\langle \xi_{\Delta t}^{4} \right\rangle 
&= \left\langle \left( \hat{W}(t + \Delta t) - \hat{W}(t) \right)^4 \right\rangle \notag \\
&= \iint dx dx' (x' - x)^4 P(\hat{W}(t + \Delta t) = x', \hat{W}(x) = x) \notag \\
&= \iint dx dx' (x' - x)^4 P(\hat{W}(t + \Delta t) = x' | \hat{W}(t) = x) P(\hat{W}(t) = x) \notag \\
& \quad (\because \omega \equiv x' - x,\ dx' = d\omega) \notag \\
&= \int dx \int d\omega \, \omega^4 \frac{1}{\sqrt{2\pi \Delta t}} e^{-\omega^2/(2\Delta t)} P(\hat{W}(t) = x) \notag \\
&= \int d\omega \, \omega^4 \frac{1}{\sqrt{2\pi \Delta t}} e^{-\omega^2/(2\Delta t)} \quad (\because \text{規格化})\notag \\
&= \frac{1}{\sqrt{2\pi \Delta t}} \cdot \frac{3}{4} \sqrt{(2\Delta t)^5 \pi} \quad (\because \text{ガウス積分}) \notag \\
&= 3 \Delta t^2
\end{align}

また、$\xi_{\Delta t}$ の 2 次のモーメントは、
\begin{align}
\left\langle \xi_{\Delta t}^2 \right\rangle 
&= \int d\omega \, \omega^2 \frac{1}{\sqrt{2\pi \Delta t}} e^{-\omega^2/(2\Delta t)} \notag \\
&= \frac{1}{\sqrt{2\pi \Delta t}} \cdot \frac{1}{2} \sqrt{(2\Delta t)^3 \pi} \notag \\
&= \Delta t
\end{align}

よって、
\begin{align}
\left\langle \left( \xi_{\Delta t}^2 - \Delta t \right)^2 \right\rangle 
&= \left\langle \xi_{\Delta t}^4 - 2\xi_{\Delta t}^2 \Delta t + (\Delta t)^2 \right\rangle \notag \\
&= 3 (\Delta t)^2 - 2 (\Delta t)^2 + (\Delta t)^2 \notag \\
&= 2 (\Delta t)^2
\end{align}

また、
\begin{align}
  D_{\Delta t} := \sum_{n=0}^{N-1} f(n \Delta t) \Delta t - \int_0^{\tau} dt f(t)
\end{align}
とする。このとき、
\begin{align}
&\lim_{\Delta t \to 0} 
\left\langle 
\left( 
\sum_{n=0}^{N-1} (\hat{\xi}_{\Delta t}(n\Delta t))^2 f(n\Delta t) 
- \int_0^{\tau} dt f(t) 
\right)^2 
\right\rangle
\\
&= \lim_{\Delta t \to 0} 
\left\langle 
\left( 
\sum_{n=0}^{N-1} \left( \hat{\xi}_{\Delta t}(n\Delta t)^2 - \Delta t \right) f(n\Delta t) + D_{\Delta t}
\right)^2 
\right\rangle \notag \\
&\quad (\because \langle \hat{\xi}_{\Delta t}(n\Delta t) \hat{\xi}_{\Delta t}(n'\Delta t) \rangle = 0 \quad (n \ne n')) \notag \\
&= \lim_{\Delta t \to 0} 
\sum_{n=0}^{N-1} 
\left\langle 
\left( \hat{\xi}_{\Delta t}(n\Delta t)^2 - \Delta t \right)^2 
\right\rangle 
f(n\Delta t)^2 + O(D_{\Delta t}) \notag \\
&= \lim_{\Delta t \to 0} 
2 (\Delta t)^2 \sum_{n=0}^{N-1} f(n\Delta t)^2 + O(D_{\Delta t}) \notag \\
&= \lim_{\Delta t \to 0} 
O(\Delta t^2) \times O(\Delta t^{-1}) + O(D_{\Delta t}) \notag \\
&= \lim_{\Delta t \to 0} 
O(\Delta t) + O(D_{\Delta t}) \notag \\
&= 0
\end{align}
がいえる。

また、同様にして、
\begin{align}
\lim_{\Delta t \to 0} 
\left\langle 
\left( \sum_{n=0}^{N-1} \hat{\xi}_{\Delta t}(n \Delta t) \Delta t f(n \Delta t) \right)^2 
\right\rangle 
&= \lim_{\Delta t \to 0} 
\sum_{n=0}^{N-1} 
\left\langle \hat{\xi}_{\Delta t}(n \Delta t)^2 \right\rangle 
(\Delta t)^2 f(n \Delta t)^2 \\
&= \lim_{\Delta t \to 0} 
O(\Delta t^3) \times O(\Delta t^{-1})  \\
&= 0 
\end{align}
および
\begin{align}
\lim_{\Delta t \to 0}
\left\langle
\left( \sum_{n=0}^{N-1} \hat{\xi}_{\Delta t}(n \Delta t)^k f(n \Delta t) \right)^2
\right\rangle
&= \lim_{\Delta t \to 0}
\sum_{n=0}^{N-1}
\left\langle \hat{\xi}_{\Delta t}(n \Delta t)^k \right\rangle f(n \Delta t)^2 \\
&= \lim_{\Delta t \to 0}
O(\Delta t^{k/2}) \times O(\Delta t^{-1}) \\
&= 0
\end{align}
が成り立つ。\qed\\

\subsubsection{確率微分方程式と確率積分}

\end{document}