\documentclass[a4paper,11pt]{jsarticle}

% 数式
\usepackage{amsmath,amsfonts}
\usepackage{amsthm}
\usepackage{bm}
\usepackage{mathtools}
\usepackage{amssymb}

% 表
\usepackage[utf8]{inputenc}
\usepackage{diagbox} % 斜線付きセルを作成するために必要
\usepackage{booktabs} % 表の罫線を美しくするために必要
\usepackage{hhline} % 水平罫線を制御するために必要

% 画像
\usepackage[dvipdfmx]{graphicx}
\usepackage{ascmac}
\usepackage{physics}
\usepackage{float} % 追加

% 図
\usepackage[dvipdfmx]{graphicx}
\usepackage{tikz} %図を描く
\usetikzlibrary{positioning, intersections, calc, arrows.meta,math} %tikzのlibrary

% ハイパーリンク
\usepackage[dvipdfm,
  colorlinks=false,
  bookmarks=true,
  bookmarksnumbered=false,
  pdfborder={0 0 0},
  bookmarkstype=toc]{hyperref}

% 式番号を章ごとにリセット
\numberwithin{equation}{section}

\begin{document}

\title{B4ゼミ\#5}
\author{大上由人}
\date{\today}
\maketitle

\setcounter{section}{3}
\section{連続空間における確率過程}
前節では、離散状態に対する確率過程について述べたが、今回は連続状態の確率過程を考える。
\subsection{数学的基礎}
\subsubsection{Wiener過程}
標準的な連続空間のMarkov過程として、Wiener過程が知られている。
\begin{itembox}[l]{\textbf{Def.Wiener過程}}
    Wiener過程$\hat{W}(t)$は、
    \begin{align}
        P(\hat{W}(t+\Delta t)=x \mid \hat{W}(t)=x')& = \frac{1}{\sqrt{2\pi \Delta t}} \exp(-\frac{(x-x')^2}{2\Delta t})\\
        P(x,0)&=\delta(x)
    \end{align}
    を満たす確率過程である。
\end{itembox}
Wiener過程が確かに存在することを直感的に確かめる。一次元格子におけるランダムウォークを考える。格子定数を$a$、時間間隔を$\Delta \tau$とする。
各回のステップでは、確率$1/2$で右に$a$、$1/2$で左に$a$移動するものとする。すなわち、
\begin{align}
    T_{x \to x+a} = T_{x \to x-a} = \frac{1}{2}
\end{align}
とする。初期条件を$P(x,0)=\delta_{x,0}$とすると、時刻$t$における$x$の期待値は常に0である。
このとき、拡散定数は、
\begin{align}
    D &:= \frac{\dd{\langle x^2 \rangle}}{\dd{t}} \\
  &= \sum_{x} \frac{ x^2 P(x,t + \Delta t) - P(x,t)}{\Delta t} \\
  &= \sum_{x} \frac{x^2}{\Delta t} \left[ P_{x-a \rightarrow x} P(x - a, t) + P_{x + a \rightarrow x} P(x + a, t) - P(x,t) \right] \\
  &= \sum_{x} \frac{x^2}{\Delta t} \left[ \frac{1}{2} P(x - a, t) + \frac{1}{2} P(x + a, t) - P(x,t) \right] \\
  &= \frac{1}{\Delta t} \left[ \frac{1}{2} \sum_{x} x^2 P(x - a, t) + \frac{1}{2} \sum_{x} x^2 P(x + a, t) - \sum_{x} x^2 P(x,t) \right] \\
  &= \frac{1}{\Delta t} \left[ \frac{1}{2} \sum_{x'} (x' + a)^2 P(x', t) + \frac{1}{2} \sum_{x''} (x'' - a)^2 P(x'', t) - \sum_{x} x^2 P(x,t) \right] \\
  &= \frac{1}{\Delta t} \sum_{x} \left[ \frac{(x - a)^2 + (x + a)^2}{2} - x^2 \right] P(x,t) \\
  &= \frac{a^2}{\Delta t} 
\end{align}

このとき、確率分布は中心極限定理によりガウス分布に収束する。
\begin{itembox}[l]{\textbf{Recall.中心極限定理}}
  $n$個の独立同一分布の確率変数$X_i$の平均値は、$n \to \infty$で、平均$\mu$、分散$\sigma^2/n$の正規分布に従う。
  ただし、$\mu$は$X_i$の平均、$\sigma^2$は$X_i$の分散である。

\end{itembox}
同じことであるが、$S_n = \sum_{i=1}^n X_i$とすると、$S_n$は平均$\mu n$、分散$\sigma^2 n$の正規分布に従う。
これを用いると、
\begin{align}
  P(x, t) 
  &= \frac{1}{\sqrt{2\pi a^2 n}} \exp\left[ -\frac{x^2}{2 a^2 n} \right] \\
  &\because a^2 / \Delta t = 1,\ t = n \Delta t \\
  &\Rightarrow a^2 n = t \\
  &= \frac{1}{\sqrt{2\pi t}} \exp\left[ -\frac{x^2}{2t} \right]
\end{align}
が得られる。
また,
\begin{align}
  P(x, t + \Delta t | x', t) 
  &= P(x, \Delta t | x', 0) \quad \text{(Markov性)} \\
  &=  P(x - x', \Delta t | 0, 0) \quad \text{(並進不変性)} \\
  &=\int \dd{x''}\, P(x - x', \Delta t | x'', 0)\, \delta(x'')  \\
  &= \int \dd{x''}\, P(x - x', \Delta t | x'', 0)\, P(x'', 0) \quad \text{(初期条件)} \\
  &= P(x - x', \Delta t) \\
  &= \frac{1}{\sqrt{2\pi \Delta t}} \exp\left[ -\frac{(x - x')^2}{2 \Delta t} \right]
\end{align}
となり、Wiener過程の定義式と一致する。
上の構成から、Wiener過程は、ランダムウォークを粗視化したときに得られるものであることがわかる。\\

Wiener過程の極限として白色Gaussノイズを導入する。
\begin{itembox}[l]{\textbf{Def.白色Gaussノイズ}}
    白色Gaussノイズは、以下のように与えられる。
    \begin{align}
        \hat{\xi}(t) = \lim_{\Delta t \to 0} \frac{\hat{W}(t+\Delta t) - \hat{W}(t)}{\Delta t}
    \end{align}
\end{itembox}
また、後の便宜のため、時間$\Delta t$の間で離散化した白色ガウスノイズを
\begin{align}
    \hat{\xi}_{\Delta t}(t) \coloneq \hat{W}(t+\Delta t) - \hat{W}(t)
\end{align}
と定義する。
このとき、ノイズの定義から
\begin{align}
    \ev{\hat{\xi}(t)} &= 0 \label{eq:xi_mean} \\
    \ev{\hat{\xi}(t) \hat{\xi}(t')} &= 0 \quad (t \neq t') \label{eq:xi_correlation} \\
\end{align}
が成り立つ。第一式は、Wiener過程の期待値が0であることと期待値の線形性から従う。
また、Markov性により時間相関がないことから第二式が成り立つ。

また、Wiener過程を
\begin{align}
    \hat{W}(\tau) = \int_0^{\tau} \dd{t} \hat{\xi}(t)
\end{align}
と復元できる。上記の関係を、正式に、
\begin{align}
    \dd{\hat{W}(t)} = \hat{\xi}(t) \dd{t}
\end{align}

(\ref{eq:xi_correlation})より、より、$\ev{\hat{\xi}(t) \hat{\xi}(t')} = k\delta(t-t')$と仮定する。
以下、この係数$k$の値を決める。
\begin{align}
  \ev{\hat{W}^2(\tau)} &= \int \dd{x}x^2 P(\hat{W}(\tau)=x) \\
  &= \int \dd{x}x^2 \frac{1}{\sqrt{2\pi \tau}} e^{-\frac{x^2}{2\tau}} \\
  &= \frac{1}{\sqrt{2\pi \tau}} \cdot \frac{1}{2} \sqrt{(2\tau)^3 \pi} \\
  &= \tau
\end{align}
また、
\begin{align}
  \ev{\hat{W}^2(\tau)} &= \int_0^{\tau} \dd{t} \int_0^{\tau} \dd{t'} \ev{\hat{\xi}(t) \hat{\xi}(t')} \\
  &= \int_0^{\tau} \dd{t} \int_0^{\tau} \dd{t'} k\delta(t-t') \\
  &= k\tau
\end{align}
よって、$k=1$となる。すなわち、
\begin{align}
    \ev{\hat{\xi}(t) \hat{\xi}(t')} = \delta(t-t')
\end{align}
が成り立つ。\\


特に、Wiener過程の二乗はアンサンブル平均をとることなく
\begin{align}
    (\dd{\hat{W}(t)})^2 = \dd{t}
\end{align}
を満たす。正確には、$(\dd{\hat{W}(t)})^2$による積分は、二乗平均の極限の意味で、通常の時間積分と同等である。
  \footnote{
\(\lim_{\Delta t \to 0} g_{\Delta t} = g\) が平均二乗収束の意味で成り立つとは、  
\[
\lim_{\Delta t \to 0} \left\langle (g_{\Delta t} - g)^2 \right\rangle = 0
\]
が成り立つことをいう。
}
\begin{itembox}[l]{\textbf{Thm.Ito則}}
  \begin{align}
    \lim_{\Delta t \to 0} \left\langle
      \left( \sum_{n=0}^{N-1} \hat{\xi}_{\Delta t}(n\Delta t)^2 f(n\Delta t)
      - \int_0^{\tau} \dd{t} f(t) \right)^2
    \right\rangle &= 0, \\
    \lim_{\Delta t \to 0} \left\langle
      \left( \sum_{n=0}^{N-1} \hat{\xi}_{\Delta t}(n\Delta t) \Delta t f(n\Delta t)
      \right)^2
    \right\rangle &= 0, \\
    \lim_{\Delta t \to 0} \left\langle
      \left( \sum_{n=0}^{N-1} \hat{\xi}_{\Delta t}(n\Delta t)^k f(n\Delta t)
      \right)^2
    \right\rangle &= 0,
  \end{align}
      ここで \( N := \tau / \Delta t \), \( k \geq 3 \) とする。これらの関係式は形式的には次のように書ける。
      
      \begin{align}
        (\dd{\hat{W}(t)})^2 &= \dd{t}, \\
        \dd{\hat{W}(t)} \dd{t} = (\dd{\hat{W}(t)})^k &= 0.
      \end{align}
      
\end{itembox}
\textbf{Prf.}\\
$\xi_{\Delta t}$ の 4 次のモーメントは、
\begin{align}
\left\langle \xi_{\Delta t}^{4} \right\rangle 
&= \left\langle \left( \hat{W}(t + \Delta t) - \hat{W}(t) \right)^4 \right\rangle \notag \\
&= \iint \dd{x} \dd{x'} (x' - x)^4 P(\hat{W}(t + \Delta t) = x', \hat{W}(t) = x) \notag \\
&= \iint \dd{x} \dd{x'} (x' - x)^4 P(\hat{W}(t + \Delta t) = x' | \hat{W}(t) = x) P(\hat{W}(t) = x) \notag \\
& \quad (\because \omega \equiv x' - x,\ \dd{x'} = \dd{\omega}) \notag \\
&= \int \dd{x} \int \dd{\omega} \, \omega^4 \frac{1}{\sqrt{2\pi \Delta t}} e^{-\omega^2/(2\Delta t)} P(\hat{W}(t) = x) \notag \\
&= \int \dd{\omega} \, \omega^4 \frac{1}{\sqrt{2\pi \Delta t}} e^{-\omega^2/(2\Delta t)} \quad (\because \text{規格化})\notag \\
&= \frac{1}{\sqrt{2\pi \Delta t}} \cdot \frac{3}{4} \sqrt{(2\Delta t)^5 \pi} \quad (\because \text{ガウス積分}) \notag \\
&= 3 \Delta t^2
\end{align}

また、$\xi_{\Delta t}$ の 2 次のモーメントは、
\begin{align}
\left\langle \xi_{\Delta t}^2 \right\rangle 
&= \int \dd{\omega} \, \omega^2 \frac{1}{\sqrt{2\pi \Delta t}} e^{-\omega^2/(2\Delta t)} \notag \\
&= \frac{1}{\sqrt{2\pi \Delta t}} \cdot \frac{1}{2} \sqrt{(2\Delta t)^3 \pi} \notag \\
&= \Delta t
\end{align}

よって、
\begin{align}
\left\langle \left( \xi_{\Delta t}^2 - \Delta t \right)^2 \right\rangle 
&= \left\langle \xi_{\Delta t}^4 - 2\xi_{\Delta t}^2 \Delta t + (\Delta t)^2 \right\rangle \notag \\
&= 3 (\Delta t)^2 - 2 (\Delta t)^2 + (\Delta t)^2 \notag \\
&= 2 (\Delta t)^2
\end{align}

また、
\begin{align}
  D_{\Delta t} := \sum_{n=0}^{N-1} f(n \Delta t) \Delta t - \int_0^{\tau} \dd{t} f(t) \label{eq:D_delta}
\end{align}
とする。このとき、
\begin{align}
&\lim_{\Delta t \to 0} 
\left\langle 
\left( 
\sum_{n=0}^{N-1} (\hat{\xi}_{\Delta t}(n\Delta t))^2 f(n\Delta t) 
- \int_0^{\tau} \dd{t} f(t) 
\right)^2 
\right\rangle
\\
&= \lim_{\Delta t \to 0} 
\left\langle 
\left( 
\sum_{n=0}^{N-1} \left( \hat{\xi}_{\Delta t}(n\Delta t)^2 - \Delta t \right) f(n\Delta t) + D_{\Delta t}
\right)^2 
\right\rangle \notag \\
&\quad (\because \langle \hat{\xi}_{\Delta t}(n\Delta t) \hat{\xi}_{\Delta t}(n'\Delta t) \rangle = 0 \quad (n \ne n')) \notag \\
&= \lim_{\Delta t \to 0} 
\sum_{n=0}^{N-1} 
\left\langle 
\left( \hat{\xi}_{\Delta t}(n\Delta t)^2 - \Delta t \right)^2 
\right\rangle 
f(n\Delta t)^2 + O(D_{\Delta t}) \notag \\
&= \lim_{\Delta t \to 0} 
2 (\Delta t)^2 \sum_{n=0}^{N-1} f(n\Delta t)^2 + O(D_{\Delta t}) \notag \\
&= \lim_{\Delta t \to 0} 
O(\Delta t) + O(D_{\Delta t}) \notag \\
&= 0
\end{align}
がいえる。\footnote{
  $N\Delta t = \tau$を用いた。
}

また、同様にして、
\begin{align}
\lim_{\Delta t \to 0} 
\left\langle 
\left( \sum_{n=0}^{N-1} \hat{\xi}_{\Delta t}(n \Delta t) \Delta t f(n \Delta t) \right)^2 
\right\rangle 
&= \lim_{\Delta t \to 0} 
\sum_{n=0}^{N-1} 
\left\langle \hat{\xi}_{\Delta t}(n \Delta t)^2 \right\rangle 
(\Delta t)^2 f(n \Delta t)^2 \\
&= \lim_{\Delta t \to 0} 
O(\Delta t^2)\\
&= 0 
\end{align}
および
\begin{align}
\lim_{\Delta t \to 0}
\left\langle
\left( \sum_{n=0}^{N-1} \hat{\xi}_{\Delta t}(n \Delta t)^k f(n \Delta t) \right)^2
\right\rangle
&= \lim_{\Delta t \to 0}
\sum_{n=0}^{N-1}
\left\langle \hat{\xi}_{\Delta t}(n \Delta t)^k \right\rangle f(n \Delta t)^2 \\
&= \lim_{\Delta t \to 0}
O(\Delta t^{k/2-1})\\
&= 0
\end{align}
が成り立つ。\qed\\

% この教科書の以降では、単純な2乗 \((\hat{\xi}_{\Delta \tau}(t))^2\) を意味するのではなく以下のように形式的に \(\hat{\xi}_{\Delta \tau}^2(t)\) と記述する。
% \begin{align}
% \hat{\xi}_{\Delta \tau}^2(t) := \sum_{n=0}^{N_{\Delta \tau} - 1} \left( \hat{\xi}_{\Delta t}(t + n\Delta t) \right)^2
% \end{align}
% ここで \(N_{\Delta \tau} := \Delta \tau / \Delta t\) は、アンサンブル平均を取ることなく \(\Delta \tau\) に収束する。
% %TODO: 上のやつは後で確認する。
\subsubsection{確率微分方程式と確率積分}
\textbf{特異点}\\
Wiener過程における微分や積分について考える。
Wiener過程はほとんどいたるところで特異点を持つため、取り扱いに注意が必要である。
$f(t,\hat{W}(t))$のような、Wiener過程に陽に依存する量を考える。
このとき、$f(t,\hat{W}(t))$の$\hat{W}(t)$に関する積分は以下のように定義される。
\begin{align}
\int_0^\tau \dd{\hat{W}(t)}\, f(t, \hat{W}(t)) := \lim_{\Delta t \to 0} \sum_{n=0}^{N-1} \hat{\xi}_{\Delta t}(t_n) f(t_n, \hat{W}_{\Delta t}(t'_n)) 
\end{align}
ここで、$t_n = n\Delta t$とし、$t_{n}'=t_{n} + \alpha \Delta t\quad (0\leq \alpha \leq 1)$とする。
現時点では、$\alpha$の任意性を残しておく。また、離散化したWiener過程を以下のように定義する。

\begin{align}
\hat{W}_{\Delta t}(t_n) &:= \sum_{m=0}^{n-1} \hat{\xi}_{\Delta t}(t_m) \\
\hat{W}_{\Delta t}(t'_n) &:= \hat{W}_{\Delta t}(t_n) + \alpha \hat{\xi}_{\Delta t}(t_n)
\end{align}
ここで、どのように$\alpha$を選ぶかの問題が生じる。
普通の積分では、$\Delta t \to 0$の極限をとれば、$\alpha$の値に関係なく一意に積分が定まる。
しかし、Wiener過程の積分では、$\alpha$の値によって積分結果が異なってしまう。
これを見るために、$f(t,\hat{W}(t)) = \hat{W}(t)$としてみる。このとき、
\begin{align}
\left\langle \int_0^\tau \dd{\hat{W}(t)} \hat{W}(t) \right\rangle 
= \lim_{\Delta t \to 0} \sum_{n=0}^{N-1} \ev{\hat{\xi}_{\Delta t}(t_n) \left( \hat{W}_{\Delta t}(t_n) + \alpha \hat{\xi}_{\Delta t}(t_n) \right)}
= \alpha \tau
\end{align}
となり、明らかに$\alpha$の値によって積分結果が異なる。\footnote{第一項が消えることは、$\hat{W}_{\Delta t}(t_n)$が$\hat{\xi}_{\Delta t}(t_n)$にの項を持たないことから従う。和の範囲を見よ。}

この性質は、確率過程であることに起因するわけではなく、あくまでも特異性に起因することに注意する。
例えば、以下のような決定論的な微分方程式を考えてみる。

\begin{align}
\dv{x}{t} = x(t)\delta(t - 1)
\end{align}
初期条件 $x(0) = 1$ のもとで、デルタ関数の積には任意性が現れる。
$t = 1$ における積の取り扱いが時間発展に影響を与え、
\begin{align}
\lim_{t \to 1} x(t)\delta(t - 1)
= (1 - \alpha) \left[ \lim_{t \to 1 - 0} x(t) \right] \delta(t - 1)
+ \alpha \left[ \lim_{t \to 1 + 0} x(t) \right] \delta(t - 1)
\end{align}
となる。これを解くことで、
\begin{align}
  x(t) = 1 + \frac{1}{1 - \alpha} \quad (t > 1)
\end{align}
となり、$\alpha$の値によって解が異なることがわかる。\\

\textbf{Ito積とStratonovich積}\\
二つの重要な積のルールを導入する。
すなわち、$\alpha$の値を定めてみる。
初めは$\alpha = 0$に対応するものを考える。
これはIto積と呼ばれる。
\begin{itembox}[l]{\textbf{Def.Ito積}}
    伊藤積分(Itô積分)は、積の記号 “\(\cdot\)” を用いて以下のように定義される。
    \begin{align}
    \int_0^\tau \dd{\hat{W}(t)} \cdot f(t, \hat{W}(t)) := \lim_{\Delta t \to 0} \sum_{n=0}^{N-1} \hat{\xi}_{\Delta t}(t_n) f(t_n, \hat{W}_{\Delta t}(t_n))
    \end{align}
    \end{itembox}
    また、$\alpha =1/2$に対応するものを考える。
    これはStratonovich積と呼ばれる。
    \begin{itembox}[l]{\textbf{Def.Stratonovich積}}
      Stratonovich積分は、積の記号 “\(\circ\)” を用いて、以下のように定義される。

    \begin{align}
    \int_0^\tau \dd{\hat{W}(t)} \circ f(t, \hat{W}(t)) := \lim_{\Delta t \to 0} \sum_{n=0}^{N-1} \hat{\xi}_{\Delta t}(t_n) f\left( t_n, \hat{W}_{\Delta t} \left( \frac{t_n + t_{n+1}}{2} \right) \right)
    \end{align}
    \end{itembox}
Ito積の定義は時間順序に重点を置いている。
というのも、新しいノイズ$\hat{\xi}_{\Delta t}(t)$が生成され、$f$に段階的に作用する。
ただし、$\hat{\xi}_{\Delta t}(t)$はこのノイズが生成される直前の関数に作用する。
言い換えれば、Ito積によって与えられる確率過程$\hat{X}(t) = \int_{0}^{\tau} \dd \hat{W}(t) \cdot f(t, \hat{W}(t))$はマルチンゲールである。
\footnote{期待値が時間に依存しないということ。}
一方、Stratonovich積は平均に重点を置いている。
すなわち、ノイズ$\hat{\xi}_{\Delta t}(t)$は、$f$に作用する直前と直後の平均に作用する。

Ito積とStratonovich積は相互変換可能である。実際、
\begin{align}
&\quad \hat{\xi}_{\Delta t}(t_n) f\left( t_n, \hat{W}_{\Delta t} \left( \frac{t_n + t_{n+1}}{2} \right) \right)\\
&= \hat{\xi}_{\Delta t}(t_n) f\left( t_n, \hat{W}_{\Delta t}(t_n) + \frac{\Delta t}{2} \pdv{\hat{W}(t_{n})}{t}  (t_n) \right)  \\
&= \hat{\xi}_{\Delta t}(t_n) f\left( t_n, \hat{W}_{\Delta t}(t_n) + \frac{1}{2} \hat{\xi}_{\Delta t}(t_n) \right) \\
&= \hat{\xi}_{\Delta t}(t_n) \left[ f(t_n, \hat{W}_{\Delta t}(t_n)) + \frac{1}{2} \hat{\xi}_{\Delta t}(t_n) \pdv{f(t_n, W)}{W} \biggr|_{W = \hat{W}_{\Delta t}(t_n)} + o(\Delta t) \right] \\
&= \hat{\xi}_{\Delta t}(t_n) f(t_n, \hat{W}_{\Delta t}(t_n)) + \frac{1}{2} \hat{\xi}_{\Delta t}^2(t_n) \pdv{f(t_n, W)}{W} \biggr|_{W = \hat{W}_{\Delta t}(t_n)} + o(\Delta t) \\
&= \hat{\xi}_{\Delta t}(t_n) f(t_n, \hat{W}_{\Delta t}(t_n)) + \frac{1}{2} \Delta t \pdv{f(t_n, W)}{W} \biggr|_{W = \hat{W}_{\Delta t}(t_n)} + o(\Delta t)
\end{align}

\(\Delta t \to 0,\ N \to \infty\) の極限を取ると、

\begin{align}
\dd\hat{W}(t) \circ f(t, \hat{W}(t)) = \dd\hat{W}(t) \cdot f(t, \hat{W}(t)) + \frac{1}{2} \dd t \pdv{f(t, W)}{W} \biggr|_{W = \hat{W}(t)}
\end{align}
となる。

Ito積が持つ重要な性質として、非予測性が挙げられる。
\begin{itembox}[l]{\textbf{Thm.Ito積の非予測性}}
  \begin{align}
\left\langle \int_0^\tau \dd{\hat{W}(t)} \cdot f(t, \hat{W}(t)) \right\rangle
= \int_0^\tau \left\langle \dd{\hat{W}(t)} \right\rangle \left\langle f(t, \hat{W}(t)) \right\rangle = 0
\end{align}
\end{itembox}
\textbf{Prf.}\\
1つ目の等式は、$f$の引数が$\hat{W}(t),t$であることと、$\dd{\hat{W}(t)}$が$\hat{W}(t),t$に依存しないことから従う。
二本目の等号は、$\hat{W}(t)$の期待値が0であることから従う。
\qed\\

また、Stratonovich積は、以下の性質を満たす。
\begin{itembox}[l]{\textbf{Thm.Stratonovich積における連鎖律}}
  \begin{align}
\dd{f(t, \hat{W}(t))} = \pdv{f}{t} \dd{t} + \pdv{f}{\hat{W}} \circ \dd{\hat{W}(t)}
\end{align}
\end{itembox}
これは、正確には以下の意味である。
\begin{align}
\Delta f(t, \hat{W}(t)) &:= f(t + \Delta t, \hat{W}_{\Delta t}(t) + \hat{\xi}_{\Delta t}(t)) - f(t, \hat{W}_{\Delta t}(t)) \notag \\
&= \pdv{f}{t} \Delta t + \pdv{f}{\hat{W}_{\Delta t}} \biggr|_{\hat{W} = \hat{W}_{\Delta t}(t) + \hat{\xi}_{\Delta t}(t)/2} \hat{\xi}_{\Delta t}(t) + o(\Delta t)
\end{align}
\textbf{Prf.}\\
\begin{align}
f(t + \Delta t, \hat{W}_{\Delta t}(t) + \hat{\xi}_{\Delta t}(t)) &=
f + \pdv{f}{t} \Delta t + \pdv{f}{\hat{W}_{\Delta t}} \biggr|_{\hat{W}_{\Delta t}(t)} \hat{\xi}_{\Delta t}(t) \notag \\
&\quad + \frac{1}{2} \pdv[2]{f}{\hat{W}_{\Delta t}} \biggr|_{\hat{W}_{\Delta t}(t)} \hat{\xi}_{\Delta t}(t)^2 + o(\Delta t) \\
&= f + \pdv{f}{t} \Delta t + \pdv{f}{\hat{W}_{\Delta t}} \biggr|_{\hat{W}_{\Delta t}(t) + \hat{\xi}_{\Delta t}(t)/2} \hat{\xi}_{\Delta t}(t) + o(\Delta t)\\
\because \text{Taylor展開の逆} \notag
\end{align}
\qed\\

一方で、Stratonovich積分は非予測性を満たさない。

\begin{align}
\left\langle \int_0^\tau \dd{\hat{W}(t)} \circ f(t, \hat{W}(t)) \right\rangle \ne 0
\end{align}

また、Ito積は連鎖律を満たさない。
\begin{align}
\dd{f(t, \hat{W}(t))} \ne \pdv{f}{t} \dd{t} + \pdv{f}{\hat{W}} \cdot \dd{\hat{W}(t)}
\end{align}
Ito積分への連鎖律の拡張はItoの補題と呼ばれ、後の章で述べる。


\end{document}