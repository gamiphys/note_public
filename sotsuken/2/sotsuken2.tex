\documentclass[a4paper,11pt]{jsarticle}

% 数式
\usepackage{amsmath,amsfonts}
\usepackage{amsthm}
\usepackage{bm}
\usepackage{mathtools}
\usepackage{amssymb}

% 表
\usepackage[utf8]{inputenc}
\usepackage{diagbox} % 斜線付きセルを作成するために必要
\usepackage{booktabs} % 表の罫線を美しくするために必要
\usepackage{hhline} % 水平罫線を制御するために必要

% 画像
\usepackage[dvipdfmx]{graphicx}
\usepackage{ascmac}
\usepackage{physics}
\usepackage{float} % 追加

% 図
\usepackage[dvipdfmx]{graphicx}
\usepackage{tikz} %図を描く
\usetikzlibrary{positioning, intersections, calc, arrows.meta,math} %tikzのlibrary

% ハイパーリンク
\usepackage[dvipdfm,
  colorlinks=false,
  bookmarks=true,
  bookmarksnumbered=false,
  pdfborder={0 0 0},
  bookmarkstype=toc]{hyperref}

% 式番号を章ごとにリセット
\numberwithin{equation}{section}

% コマンド定義
\newcommand{\argmin}{\mathop{\mathrm{arg\,min}}\limits}
\newcommand{\argmax}{\mathop{\mathrm{arg\,max}}\limits}

\begin{document}

\title{Cross asymmetry}
\author{大上由人}
\date{\today}
\maketitle

\section{モチベーション}
$a(t),b(t)$をそれぞれ時刻$t$における物理量とする。
このとき、cross-correlationを
\begin{align}
    C_{ba}^{\tau} &= \ev{b(t+\tau)a(t)}
\end{align}
で定義する。系が奇変数を持たないときを考える。
このとき、平衡状態ならば、
\begin{align}
    C_{ab}^{\tau} &= C_{ba}^{\tau}
\end{align}
が成り立つことが知られている。
しかし、非平衡定常状態においては一般に
\begin{align}
    C_{ab}^{\tau} &\neq C_{ba}^{\tau}
\end{align}
であることが知られている。
このことを踏まえると、cross-correlationの非対称性は系の非平衡性度合いを示す指標となりうる。

一般に非平衡定常状態を維持するためにはThermodynamic forceが必要である(温度勾配、化学ポテンシャル勾配など)。
このことを踏まえると、\textbf{cross-correlationの非対称性とThermodynamic forceの間には何らかの関係がある}と予想される。

\section{準備}
以下、Markov jump processを考える。
確率分布を$\vb{p} = (p_1, p_2, \ldots, p_n)^T$と表す。
これが従う
マスター方程式は、
\begin{align}
    \dv{\vb{p}}{t} &=R\vb{p}
\end{align}
である。ただし$R$は遷移レートである。
また、特に定常状態を$\vb{q}$と表す。
これは、
\begin{align}
    R\vb{q} &=0
\end{align}
を満たす。定常状態における一方向確率流を
\begin{align}
    \mathcal{T}_{ij} &:= R_{ij}q_j
\end{align}
と定義する。


また、状態を頂点としてグラフを作ることを考える。
状態間を結んでできるサイクルを$c$と表す。
単純サイクル$c= (i_1,i_2,\ldots,i_n)$に注目する。
以下、$R_{i_{k+1}i_k} >0$が成り立つことを仮定する。
このとき、サイクル$c$に沿ったThermodynamic forceを
\begin{align}
    \mathcal{F}_c   &= \ln \frac{R_{i_2 i_1} R_{i_3 i_2} \cdots R_{i_1 i_n}}{R_{i_1 i_2} R_{i_2 i_3} \cdots R_{i_n i_1}}
\end{align}
により定義する。これは1サイクルに沿った熱浴のエントロピー変化に対応する。\\

非対称性を測る無次元量の指標として以下の量を導入する:
\begin{align}
    \chi_{ba} := \lim_{\tau \to 0} \frac{C_{ba}^{\tau} - C_{ab}^{\tau}}{2\sqrt{(\Delta_{\tau}C_{aa})(\Delta_{\tau} C_{bb})}}
\end{align}
ただし、
\begin{align}
    \Delta_{\tau} C_{aa} &:= C_{aa}^{0} - C_{aa}^{\tau} 
\end{align}
であり、これは自己相関の減少を表す。
% また、これは拡散の尺度になることも知られている(確かめなきゃ)。
また、$\chi_{ba}$はスケール不変である。
\footnote{
$a'(t) = \alpha a(t), b'(t) = \beta b(t)$としたとき、
\begin{align}
C_{b'a'}^{\tau} &= \langle b'(t+\tau)a'(t) \rangle = \langle (\beta b(t+\tau)) (\alpha a(t)) \rangle = \alpha\beta \langle b(t+\tau)a(t) \rangle = \alpha\beta C_{ba}^{\tau} \\
C_{a'b'}^{\tau} &= \langle a'(t+\tau)b'(t) \rangle = \langle (\alpha a(t+\tau)) (\beta b(t)) \rangle = \alpha\beta \langle a(t+\tau)b(t) \rangle = \alpha\beta C_{ab}^{\tau}
\end{align}
したがって、分子は、
\begin{align}
C_{b'a'}^{\tau} - C_{a'b'}^{\tau} = \alpha\beta C_{ba}^{\tau} - \alpha\beta C_{ab}^{\tau} = \alpha\beta (C_{ba}^{\tau} - C_{ab}^{\tau})
\end{align}
となる。
また、分母は、
\begin{align}
C_{a'a'}^{\tau} &= \langle a'(t+\tau)a'(t) \rangle = \langle (\alpha a(t+\tau)) (\alpha a(t)) \rangle = \alpha^2 C_{aa}^{\tau} \\
\Delta_{\tau}C_{a'a'} &= C_{a'a'}^{0} - C_{a'a'}^{\tau} = \alpha^2 C_{aa}^{0} - \alpha^2 C_{aa}^{\tau} = \alpha^2 (C_{aa}^{0} - C_{aa}^{\tau}) = \alpha^2 \Delta_{\tau}C_{aa}
\end{align}
同様にして、
\begin{align}
\Delta_{\tau}C_{b'b'} = \beta^2 \Delta_{\tau}C_{bb}
\end{align}
となる。
以上を合わせればスケール不変性が示される:
\begin{align}
\chi_{b'a'} &= \lim_{\tau\rightarrow 0} \frac{\alpha\beta (C_{ba}^{\tau} - C_{ab}^{\tau})}{2 \alpha\beta \sqrt{(\Delta_{\tau}C_{aa})(\Delta_{\tau}C_{bb})}}\\
&=\lim_{\tau\rightarrow 0} \frac{C_{ba}^{\tau} - C_{ab}^{\tau}}{2 \sqrt{(\Delta_{\tau}C_{aa})(\Delta_{\tau}C_{bb})}} \\
&= \chi_{ba}
\end{align}
また、物理量の定数シフトに対しても不変であることが示せる。
}


$\tau$が小さい時
\begin{align}
    C_{ba}^{\tau} &= \sum_{i,j}e^{\tau R}_{ij} q_j b_i a_j \\
    &\simeq \sum_{i,j}(\delta_{ij} + \tau R_{ij}) q_j b_i a_j \\
    &= \sum_{i} q_i b_i a_i + \tau \sum_{i,j} R_{ij} q_j b_i a_j
\end{align}
と書ける。このとき、
\begin{align}
    \Delta_{\tau} C_{aa} &=C_{aa}^{0} - C_{aa}^{\tau} \\
    &\simeq -\tau \sum_{i,j} \mathcal{T}_{ij} a_i a_j\\
    &= \frac{\tau}{2}\sum_{i,j}\mathcal{T}_{ij}(a_i - a_j)^2
\end{align}

と書くことができる。ただし、最後の式変形では$\sum_{i} \mathcal{T}_{ij} =0, \sum_{j} \mathcal{T}_{ij} =0$を用いた。

このとき、
\begin{align}
    \chi_{ba} &= \frac{\sum_{i,j}\mathcal{T}_{ij}(b_ia_j-b_ja_i)}{2\sqrt{-\sum_{i,j} \mathcal{T}_{ij}a_ia_j }\sqrt{-\sum_{i,j} \mathcal{T}_{ij}b_ib_j}}
\end{align}
と書くことができる。

\section{主結果}
\subsection{主結果}
以下の不等式が成り立つことが示される。
\begin{itembox}[l]{\textbf{Thm.1}}
    $\chi_{ba}$について、以下の不等式が成り立つ:
    \begin{align}
    \abs{\chi_{ba}} \leq \max_{c} 
\frac{\tanh(\mathcal{F}_c / 2n_c)}{\tan(\pi / n_c)} 
\leq \max_{c} \frac{\mathcal{F}_c}{2\pi},
\end{align}
ただし、$\max_{c}$は取りうるすべての単純サイクルについてとる。
また、$n_c$はサイクル$c$に含まれる状態の数である。
また、等号成立条件は
\begin{align}
    \mathcal{T}_{12} = \mathcal{T}_{23} &= \cdots = \mathcal{T}_{n1}\\
    \mathcal{T}_{21} = \mathcal{T}_{32} &= \cdots = \mathcal{T}_{1n}
\end{align}
および、ある$\gamma$が存在して、$(\gamma a_1, b_1), (\gamma a_2, b_2), \ldots, (\gamma a_n, b_n)$が正n角形をなすことである。
\end{itembox}
すなわち、非対称性の大きさはThermodynamic forceによって上から抑えられる。

また、これよりタイトな不等式を得ることもできる。
\begin{itembox}[l]{\textbf{Thm.2}}
    Thm.1よりタイトな不等式が成り立つ:
    \begin{align}
    \abs{\chi_{ba}}
\leq 
\max_{c \in \mathcal{C}^*}
\frac{n_c \tanh(\mathcal{F}_c / 2n_c)}{n'_c \tan(\pi / n'_c)}
\leq 
\max_{c \in \mathcal{C}^*}
\frac{\mathcal{F}_c / 2n'_c}{\tan(\pi / n'_c)}
\leq 
\max_{c \in \mathcal{C}^*}
\frac{\mathcal{F}_c}{2\pi}
\end{align}
ただし、
$n_c'$はサイクル$c$において$(a,b)$の値が変化する回数であり、$n_c' \leq n_c$である。また、
\begin{align}
    C^* &= C_{\text{asy}} \cap C_{\text{uni}} \\
    C_{\text{asy}} &:= \left\{ c \mid b_{i_{k+1}} a_{i_k} - b_{i_k} a_{i_{k+1}} \neq 0\right\} \\
    C_{\text{uni}} &= \{ c  \mid \forall e \in c, e \in \mathcal{E}^{+} \}\\
    \mathcal{E}^{+} &:= \{ e \mid \mathcal{J}_{e} = \mathcal{T}_e - \mathcal{T}_{-e} > 0 \}
\end{align}
である。
\end{itembox}
証明は後述。\\

この不等式の応用例を見ていく。
遷移レート$R$び固有値を$\lambda_1, \lambda_2, \ldots, \lambda_n$とする。
また、
\begin{align}
    \lambda_{\alpha} &= -\lambda_{\alpha}^{R} + i \lambda_{\alpha}^{I}
\end{align}
とする。
このとき、時間発展は以下のように書ける:
\begin{align}
    e^{\tau R} &= \sum_{\alpha} \exp(-\lambda_{\alpha}^{R} \tau)\exp(i \lambda_{\alpha}^{I} \tau) \vb{u}^{(\alpha)}\vb{v}^{(\alpha) T}
\end{align} 
ただし、$\vb{u}^{\alpha}, \vb{v}^{\alpha}$はそれぞれ右固有ベクトル、左固有ベクトルである。
このとき、以下の不等式が成り立つ。
\begin{itembox}[l]{\textbf{Thm.3}}
    \begin{align}
    \frac{\abs{\lambda_\alpha^{\mathrm{I}}}}{2\pi \lambda_\alpha^{\mathrm{R}}}
    \leq \max_{c} \frac{\tanh(\mathcal{F}_c / 2n_c)}{2\pi \tan(\pi / n_c)}.
\end{align}

\end{itembox}
これは振動があれば、その分だけThermodynamic forceが必要であることを示している。
\footnote{
    時間結晶との関連を考えてみると、固有値の実部が$0$に近づくわけだが、このとき左辺が発散する。
    したがって右辺も発散することとなるが、これは物理的におかしい。したがってこの系(古典・詳細つり合いあり)では時間結晶はできなさそうである。
}

\textbf{Prf.}\\
\begin{align}
    a_{i} &= \frac{\Im(u_{i}^{\alpha})}{q_i} \\
    b_{i} &= \frac{\Re(u_{i}^{\alpha})}{q_i}
\end{align}
とおく。
いま、
\begin{align}
    \sum_{i} \abs{u_{i}^{\alpha}}^2/q_i &=1
\end{align}
となるように固有ベクトルを規格化しておく。これを用いると、
\begin{align}
    \lambda = \sum_i \frac{u_i^*}{q_i} \lambda u_i
= \sum_{i,j} \frac{u_i^*}{q_i} R_{ij} u_j
= \sum_{i,j} R_{ij} q_j \frac{u_i^* u_j}{q_i q_j}.
\end{align}
と計算することができる。
ここで、$\mathcal{T}_{ij} = R_{ij} q_j$および$a_{i}= \Im(u_{i}^{\alpha})/q_i, b_{i} = \Re(u_{i}^{\alpha})/q_i$を用いることで、
\begin{align}
    \lambda &= \sum_{i,j} \mathcal{T}_{ij}(b_i - i a_i)(b_j + i a_j)\\
    &=\sum_{i,j} \mathcal{T}_{ij}(b_i b_j + a_i a_j + i b_i a_j - i a_i b_j).
\end{align}
と書くことができる。このことと、$\lambda_\alpha = -\lambda_\alpha^{R} + i \lambda_\alpha^{I}$を合わせることで、
\begin{align}
    \lambda^{\mathrm{I}}
&= \sum_{i,j} \mathcal{T}_{ij}(b_i a_j - a_i b_j)
= \lim_{\tau \to 0}
\frac{C_{ba}^{\tau} - C_{ab}^{\tau}}{\tau},\\
\lambda^{\mathrm{R}}
&= -\sum_{i,j} \mathcal{T}_{ij}(b_i b_j + a_i a_j)
= \lim_{\tau \to 0}
\frac{\Delta_\tau C_{aa} + \Delta_\tau C_{bb}}{\tau}.
\end{align}
が得られる。
これを用いると、
\begin{align}
    \abs{\chi_{ba}} &= \abs{\lim_{\tau \to 0} \frac{C_{ba}^{\tau} - C_{ab}^{\tau}}{2\sqrt{(\Delta_{\tau}C_{aa})(\Delta_{\tau} C_{bb})}}} \\
    &= \frac{\abs{(C_{ba}^{\tau} - C_{ab}^{\tau})/\tau}}{2\sqrt{(\Delta_{\tau}C_{aa}/\tau)(\Delta_{\tau} C_{bb}/\tau)}} \\
    &\geq \frac{\abs{(C_{ba}^{\tau} - C_{ab}^{\tau})/\tau}}{(\Delta_{\tau}C_{aa}+\Delta_{\tau} C_{bb})/\tau} \\
    &= \frac{\abs{\lambda_{\alpha}^{I}}}{\lambda_{\alpha}^{R}}
\end{align}
が言える。あとはThm.1の不等式を用いればよい。
\qed\\

\subsection{主結果の導出}
\subsubsection{記号のメモ}
\begin{align}
    \mathcal{J}_{ij} &:= \mathcal{T}_{ij} - \mathcal{T}_{ji} \\
    \mathcal{A}_{ij} &:= \mathcal{T}_{ij} + \mathcal{T}_{ji}\\
    \Omega_{ij} &:=\frac{1}{2}( b_i a_j - b_j a_i) \\
    L_{ij} &:= \sqrt{(a_i - a_j)^2 +(b_i - b_j)^2}
\end{align}

\subsubsection{$\chi$の書き換え}
$\chi_{ba}$のスケール不変性を利用することで、
\begin{align}
    \sum_{ij} \mathcal{T}_{ij} a_i a_j &= \sum_{ij} \mathcal{T}_{ij} b_i b_j
\end{align}
となるようにとることができる。
このことを用いて、$\chi_{ba}$を以下のように書き換える:
\begin{align}
\chi_{ba} &=
 \frac{\sum_{i,j}\mathcal{T}_{ij}(b_ia_j-b_ja_i)}{2\sqrt{-\sum_{i,j} \mathcal{T}_{ij}a_ia_j }\sqrt{-\sum_{i,j} \mathcal{T}_{ij}b_ib_j}}\\
&=
\frac{\sum_{i,j} \mathcal{T}_{ij} (b_i a_j - b_j a_i)}{- \sum_{i,j} \mathcal{T}_{ij} (a_i a_j + b_i b_j)}
= 
\frac{\sum_{i,j} \mathcal{T}_{ij} (b_i a_j - b_j a_i)}{\frac{1}{2} \sum_{i,j} \mathcal{T}_{ij} \left[ (a_i - a_j)^2 + (b_i - b_j)^2 \right]} \\
&=
\frac{4 \sum_{i,j} \mathcal{T}_{ij} \Omega_{ij}}{\sum_{i,j} \mathcal{T}_{ij} L_{ij}^2}
= 
\frac{4 \sum_{i>j} \mathcal{J}_{ij} \Omega_{ij}}{\sum_{i>j} \mathcal{A}_{ij} L_{ij}^2}.
\end{align}
ただし、2つ目の等号でスケール不変性からの条件を用いた。したがって、
\begin{align}
    \abs{\chi_{ba}} &=
    \frac{4 \abs{\sum_{i>j} \mathcal{J}_{ij} \Omega_{ij}}}{\sum_{i>j} \mathcal{A}_{ij} L_{ij}^2}.
\end{align}
と書ける。

\subsubsection{サイクルの分解}
%まじで書く
正味の流れが正であるエッジの集合を
\begin{align}
    \mathcal{E}^{+} := \{ e \mid \mathcal{J}_e > 0 \}
\end{align}
と定義する。また、$C_{\text{uni}}$を以下のように定義する:
\begin{align}
    C_{\text{uni}} = \{ c  \mid \forall e \in c, e \in \mathcal{E}^{+} \}
\end{align}
辺$e$のカレントを、$C_{\text{uni}}$に含まれる単純サイクル$c$のカレントの和として表すことができる。
すなわち、
\begin{align}
    \mathcal{J}_{e} &= \sum_{c \in C_{\text{uni}}}S_{ec} J_c \\ 
    S_{ec} &=
    \begin{cases}
        1 & \quad e \in c \\
        0 & \quad \text{otherwise}
    \end{cases}
\end{align}
と書ける。

このとき、以下が成り立つ:
\begin{align}
\sum_{e \in \mathcal{E}^+} \mathcal{J}_e X_e
&= \sum_{c \in \mathcal{C}_{\mathrm{uni}}} \mathcal{J}_c \left( \sum_{e \in \mathcal{E}^+} S_{ec} X_e \right) \\
&= \sum_{c \in \mathcal{C}_{\mathrm{uni}}} \mathcal{J}_c \left( \sum_{e \in c} X_e \right).
\end{align}
このもとで、$\chi$の分母と分子に関わる不等式をそれぞれ導いておく。
\begin{align}
\left| \sum_{i>j} \mathcal{J}_{ij} \Omega_{ij} \right|
&= \left| \sum_{e \in \mathcal{E}^+} \mathcal{J}_e \Omega_e \right| \\
&= \left| \sum_{c \in \mathcal{C}_{\mathrm{uni}}} \mathcal{J}_c \left( \sum_{e \in c} \Omega_e \right) \right| \\
&= \left| \sum_{c \in \mathcal{C}^*} \mathcal{J}_c \left( \sum_{e \in c} \Omega_e \right) \right| \\
&\leq \sum_{c \in \mathcal{C}^*} \mathcal{J}_c \left| \sum_{e \in c} \Omega_e \right|.
\end{align}
ただし、$C^* = C_{\text{asy}} \cap C_{\text{uni}}$である。また、最後に三角不等式を用いた。
また、以下も成り立つ:
\begin{align}
\sum_{i>j} \mathcal{A}_{ij} L_{ij}^2
&\geq \sum_{e \in \mathcal{E}^+} \mathcal{A}_e L_e^2 \\
&= \sum_{e \in \mathcal{E}^+} \mathcal{J}_e \frac{\mathcal{A}_e}{\mathcal{J}_e} L_e^2 \\
&= \sum_{c \in \mathcal{C}_{\mathrm{uni}}} \mathcal{J}_c \left( \sum_{e \in c} \frac{\mathcal{A}_e}{\mathcal{J}_e} L_e^2 \right) \\
&\geq \sum_{c \in \mathcal{C}^*} \mathcal{J}_c \left( \sum_{e \in c} \frac{\mathcal{A}_e}{\mathcal{J}_e} L_e^2 \right).
\end{align}

\subsection{一般化TUR}
Thermodynamic force の言葉でTUR likeな不等式を示す。
\begin{itembox}[l]{\textbf{Lem.}}
    $\forall c \in C_{\text{uni}}$に対して、
    \begin{align}
        \frac{\left( \sum_{e \in c} X_e \right)^2}{\sum_{e \in c} X_e^2 \mathcal{A}_e / \mathcal{J}_e}
\leq n_c \tanh\left( \frac{\mathcal{F}_c}{2 n_c} \right).
    \end{align}
    が成立する。
\end{itembox}
\textbf{Prf.}\\
$(X_{e}\sqrt{\mathcal{A}_e / \mathcal{J}_e} )_{e \in c}$および$(\sqrt{\mathcal{J}_e / \mathcal{A}_e}) _{e \in c}$
についてのSchwarzの不等式を用いることで、
\begin{align}
    \qty(\sum_{e \in c} X_e)^2 
&\leq \left( \sum_{e \in c} X_e^2 \frac{\mathcal{A}_e}{\mathcal{J}_e} \right) \left( \sum_{e \in c} \frac{\mathcal{J}_e}{\mathcal{A}_e} \right).\\
\Leftrightarrow \frac{\qty(\sum_{e \in c} X_e)^2}{\sum_{e \in c} X_e^2 \mathcal{A}_e / \mathcal{J}_e}
&\leq \sum_{e \in c} \frac{\mathcal{J}_e}{\mathcal{A}_e}.
\end{align}
が成り立つ。
また、Thermodynamic forceの定義から、
\begin{align}
    F_{e} &= \sum_{e \in c} \ln \frac{\mathcal{T}_e}{\mathcal{T}_{-e}} \\
    &= \sum_{e \in c} \ln \frac{\mathcal{A}_e + \mathcal{J}_e}{\mathcal{A}_e - \mathcal{J}_e}\\
    &= 2n_{c} \sum_{e \in c} \frac{1}{n_c} \mathrm{artanh} \frac{\mathcal{J}_e}{\mathcal{A}_e}
\end{align}
が成り立つ。ただし、$\text{artanh}(x) = \frac{1}{2} \ln \frac{1+x}{1-x}$である。
Jensenの不等式を用いることで、
\begin{align}
    F_{c} &\geq 2 n_c \mathrm{artanh} \left( \frac{1}{n_c} \sum_{e \in c} \frac{\mathcal{J}_e}{\mathcal{A}_e} \right).
\end{align}
が得られる。
これと、$\tanh$の単調性を用いることで、
\begin{align}
    \frac{1}{n_c} \sum_{e \in c} \frac{\mathcal{J}_e}{\mathcal{A}_e}
&\leq \tanh\left( \frac{F_c}{2 n_c} \right).
\end{align}
が得られる。
これら二つの結果をわせることで示せる。\qed\\

証明からわかるように、この結果の等号成立条件はSchwarzの不等式とJensenの不等式の等号成立条件であり、
\begin{align}
    (X_{e}\sqrt{\mathcal{A}_e / \mathcal{J}_e} )_{e \in c} &\propto (\sqrt{\mathcal{J}_e / \mathcal{A}_e}) _{e \in c}\\
    \forall e \in c, \frac{\mathcal{J}_e}{\mathcal{A}_e} &= \text{const}
\end{align}
である。

\subsection{等周不等式}
\begin{itembox}[l]{\textbf{Thm.等周不等式}}
    $\mathbb{R}^2$上の点列
    $(a_1,b_1), (a_2,b_2), \ldots, (a_n,b_n)$について、
    \begin{align}
\left( 4n \tan \frac{\pi}{n} \right)
\left| \sum_{i=1}^{n} \Omega_{i+1,i} \right|
&\leq 
\left( \sum_{i=1}^{n} L_{i+1,i} \right)^2,
\end{align}
    が成立する。等号成立条件は、$(a_1,b_1), (a_2,b_2), \ldots, (a_n,b_n)$が正n角形をなすことである。
\end{itembox}
\textbf{Prf.}\\
一旦略\qed\\

この定理から直ちに以下の不等式も示せる。
\begin{itembox}[l]{\textbf{Cor.}}
    $\mathbb{R}^2$上の点列
    $(a_1,b_1), (a_2,b_2), \ldots, (a_n,b_n)$について、
    \begin{align}
        n' &:= \left| \{ i \in \{1,2,\ldots,n\} \mid (a_{i},b_{i}) \neq (a_{i+1},b_{i+1}) \} \right|
    \end{align}
    とする。このとき、
    \begin{align}
    \left( 4n' \tan \frac{\pi}{n'} \right)
    \left| \sum_{i=1}^{n} \Omega_{i+1,i} \right|
    &\leq
    \left( \sum_{i=1}^{n} L_{i+1,i} \right)^2,
    \end{align}
\end{itembox}
\textbf{Prf.}\\
$I = \{ i \in \{1,2,\ldots,n\} \mid (a_{i},b_{i}) \neq (a_{i+1},b_{i+1}) \}$とする。
これに対する等周不等式から、
\begin{align}
\left( 4n' \tan \frac{\pi}{n'} \right)
\left| \sum_{i \in I} \Omega_{i+1,i} \right|
&\leq
\left( \sum_{i \in I} L_{i+1,i} \right)^2,
\end{align}
が得られる。
このことと、
\begin{align}
\sum_{i=1}^{n} \Omega_{i+1,i} &= \sum_{i \in I} \Omega_{i+1,i}\\
\sum_{i=1}^{n} L_{i+1,i} &= \sum_{i \in I} L_{i+1,i}.
\end{align}
を用いれば示せる。\qed\\

\subsection{主結果の証明}
%OK
主結果のタイトな場合の方を示す。初めの主結果はこの結果から直ちに従う。
\footnote{
    $n_c' \leq n_c$より、
    \begin{align}
        n_c'\tan(\pi/n_c') \geq  n_c\tan(\pi/n_c) 
    \end{align}
    が成り立つため。
}
\\
\textbf{Prf.}\\
サイクル$C = \{i_1, i_2, \ldots, i_{n_c}\}$に対して、等周不等式の系を用いることで、
\begin{align}
\left( 4n'_c \tan \frac{\pi}{n'_c} \right)
\left| \sum_{e \in c} \Omega_e \right|
&\leq
\left( \sum_{e \in c} L_e \right)^2
\end{align}た、グラフの分解のところで得られた不等式を用いることで、
\begin{align}
\left| \sum_{i>j} \mathcal{T}_{ij} \Omega_{ij} \right|
&\leq
\sum_{c \in \mathcal{C}^*} \mathcal{J}_c 
\left| \sum_{e \in c} \Omega_e \right|
\leq
\sum_{c \in \mathcal{C}^*} \mathcal{J}_c
\left( \sum_{e \in c} L_e \right)^2
\left( 4n'_c \tan \frac{\pi}{n'_c} \right)^{-1}
\end{align}
が成立する。

また、一般化TURにおいて、$X_e = L_e$とすることで、
\begin{align}
    \frac{\left( \sum_{e \in c} L_e \right)^2}{\sum_{e \in c} L_e^2 \mathcal{A}_e / \mathcal{J}_e}
&\leq n_c \tanh\left( \frac{\mathcal{F}_c}{2 n_c} \right).\\
\Leftrightarrow \qty(n_c \tanh\left( \frac{\mathcal{F}_c}{2 n_c} \right))^{-1}
\qty( \sum_{e \in c} L_e )^2
&\leq \sum_{e \in c} L_e^2 \frac{\mathcal{A}_e}{\mathcal{J}_e}
\end{align}
が得られる。さらに、グラフの分解のところで得られた不等式を用いることで、
\begin{align}
    \sum_{c \in \mathcal{C}^*} \mathcal{J}_c
\left( n_c \tanh\left( \frac{\mathcal{F}_c}{2 n_c} \right) \right)^{-1}
\left( \sum_{e \in c} L_e \right)^2
&\leq
\sum_{c \in \mathcal{C}^*} \mathcal{J}_c
\sum_{e \in c} L_e^2 \frac{\mathcal{A}_e}{\mathcal{J}_e} \\
&\leq
\sum_{i>j} \mathcal{A}_{ij} L_{ij}^2.
\end{align}
が成立する。
これら二つの不等式を用いることで、
\begin{align}
\abs{\chi_{ba}}
&\leq
\frac{
4 \sum_{c \in \mathcal{C}^*} \mathcal{J}_c (\sum_{e \in c} L_e)^2
\left[ 4n'_c \tan(\pi / n'_c) \right]^{-1}
}{
\sum_{c \in \mathcal{C}^*} \mathcal{J}_c (\sum_{e \in c} L_e)^2
\left[ n_c \tanh(\mathcal{F}_c / 2n_c) \right]^{-1}
}
\end{align}

ここで、
\begin{align}
\frac{
\sum_{c \in \mathcal{C}^*} y_c
}{
\sum_{c \in \mathcal{C}^*} x_c
}
=
\max_{c \in \mathcal{C}^*} \frac{y_c}{x_c}
\end{align}
が成り立つことを用いると、\footnote{
    $M = \max_{c \in \mathcal{C}^*} \frac{y_c}{x_c}$とする。
    このとき、
    \begin{align}
        \frac{y_c}{x_c} &\leq M \\
        \Leftrightarrow y_c &\leq M x_c\\
        \Rightarrow \sum_{c \in \mathcal{C}^*} y_c &\leq M \sum_{c \in \mathcal{C}^*} x_c \\
        \Leftrightarrow \frac{\sum_{c \in \mathcal{C}^*} y_c}{\sum_{c \in \mathcal{C}^*} x_c} &\leq M
    \end{align}
}

\begin{align}
\abs{\chi_{ba}}
&\leq
\max_{c \in \mathcal{C}^*}
\frac{
4 \mathcal{J}_c (\sum_{e \in c} L_e)^2
\left[ 4n'_c \tan(\pi / n'_c) \right]^{-1}
}{
\mathcal{J}_c (\sum_{e \in c} L_e)^2
\left[ n_c \tanh(\mathcal{F}_c / 2n_c) \right]^{-1}
}
=
\max_{c \in \mathcal{C}^*}
\frac{
n_c \tanh(\mathcal{F}_c / 2n_c)
}{
n'_c \tan(\pi / n'_c)
}
\end{align}
が示される。\qed\\

\end{document}