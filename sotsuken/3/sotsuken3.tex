\documentclass[a4paper,11pt]{jsarticle}

% 数式
\usepackage{amsmath,amsfonts}
\usepackage{amsthm}
\usepackage{bm}
\usepackage{mathtools}
\usepackage{amssymb}

% 表
\usepackage[utf8]{inputenc}
\usepackage{diagbox} % 斜線付きセルを作成するために必要
\usepackage{booktabs} % 表の罫線を美しくするために必要
\usepackage{hhline} % 水平罫線を制御するために必要

% 画像
\usepackage[dvipdfmx]{graphicx}
\usepackage{ascmac}
\usepackage{physics}
\usepackage{float} % 追加

% 図
\usepackage[dvipdfmx]{graphicx}
\usepackage{tikz} %図を描く
\usetikzlibrary{positioning, intersections, calc, arrows.meta,math} %tikzのlibrary

% ハイパーリンク
\usepackage[dvipdfm,
  colorlinks=false,
  bookmarks=true,
  bookmarksnumbered=false,
  pdfborder={0 0 0},
  bookmarkstype=toc]{hyperref}

% 式番号を章ごとにリセット
\numberwithin{equation}{section}

% コマンド定義
\newcommand{\argmin}{\mathop{\mathrm{arg\,min}}\limits}
\newcommand{\argmax}{\mathop{\mathrm{arg\,max}}\limits}

\begin{document}

\title{量子系における確率熱力学}
\author{大上由人}
\date{\today}
\maketitle

Yoshimura-Maekawa-Nagayama-Itoおよび大賀さんのトークのスライドを参考に量子系の確率熱力学をまとめる。

\section{GKSL方程式}
GKSL方程式
\begin{align}
    \dv{\rho}{t} &= -i[H,\rho] + \sum_{\mu \in \mathcal{L}_{L}}D_{\mu}[\rho] \\
    D_{\mu}[\rho] &= \sum_{\mu \in \mathcal{L}_{L}} [\gamma_{\mu}^{+} (L_{\mu} \rho L_{\mu}^{\dagger} - \frac{1}{2} \{L_{\mu}^{\dagger} L_{\mu}, \rho\}) 
    + \gamma_{\mu}^{-} (L_{\mu}^{\dagger} \rho L_{\mu} - \frac{1}{2} \{L_{\mu} L_{\mu}^{\dagger}, \rho\})]
\end{align}
である。
ただし、$\gamma_{\mu}^{\pm} \in \mathbb{R}$はパラメータ、$L_{\mu}$はjump op.である。
また、$\mu \in \mathcal{L}_{L}$はjumpの添字である。
Hamiltonianは
\begin{align}
    H = \sum_{\mu \in \mathcal{L}_{H}} (\zeta^*_{\mu} V_{\mu} + \zeta_{\mu} V_{\mu}^{\dagger})
\end{align}
と表される。
また、$\mathcal{L} = \mathcal{L}_{L} \cup \mathcal{L}_{H}$とし、
\footnote{
    便宜のため$\mathcal{L}_{L} \cap \mathcal{L}_{H} = \emptyset$とする。
}
$S_{\mu}$を
\begin{align}
    S_{\mu} =
    \begin{cases}
        L_{\mu} & (\mu \in \mathcal{L}_{L}) \\
        V_{\mu} & (\mu \in \mathcal{L}_{H})
    \end{cases}
\end{align}
と定義する。

\section{カレント}
Observable currentと呼ばれる量を考える。\\
\begin{itembox}[l]{\textbf{Def.Observable current}}
    jumpに対応するObservable currentは
    \begin{align}
        Q_{\mu}(\rho) = \gamma_{\mu}^{+} \Tr[L_{\mu}^{\dagger} L_{\mu} \rho] - \gamma_{\mu}^{-} \Tr[L_{\mu} L_{\mu}^{\dagger} \rho]
    \end{align}
    で定義される。また、Hamiltonianに対応するObservable currentは
    \begin{align}
        Q_{\mu}(\rho) = \Tr[\rho(-i\zeta_{\mu} V_{\mu}^{\dagger} + i\zeta_{\mu}^* V_{\mu})]
    \end{align}
    で定義される。また、全Observable currentは
    \begin{align}
        \vec{Q}(\rho) = \{Q_{\mu}(\rho)\}_{\mu \in \mathcal{L}_{L} \cup \mathcal{L}_{H}}
    \end{align}
    と表される。
\end{itembox}
jumpに対応するObservable currentは単位時間当たりのjump $\mu$の正味の発生回数に対応する。
また、Hamiltonianに対応するObservable currentは我々がよく知っているカレント(粒子流、電流など)に対応する。
\footnote{
    ほげほげ
}

また、Dynamical currentと呼ばれる量も考えることができる。
\begin{itembox}[l]{\textbf{Def.Dynamical current}}
    jumpに対応するDynamical currentは
    \begin{align}
        J_{\mu}(\rho) = \frac{1}{2}(\gamma_{\mu}^{+} L_{\mu}\rho - \gamma_{\mu}^{-} \rho L_{\mu})
    \end{align}
    で定義される。
    また、Hamiltonianに対応するDynamical currentは
    \begin{align}
        J_{\mu}(\rho) = i\zeta_{\mu}(\Pi_{\mu}^{+}\rho + \rho \Pi_{\mu}^{-} - \Pi_{\mu}^{+} \rho \Pi_{\mu}^{-}) 
    \end{align}
    で定義される。
    ただし、
    $\Pi_{\mu}^{+}$は$\Im V_{\mu}$への射影演算子であり、
    $\Pi_{\mu}^{-}$は$\Im V_{\mu}^{\dagger}$への射影演算子である。
\end{itembox}
これらはObservable currentよりも抽象的な定義であるが\footnote{
    定義自体はGKSL方程式での最適輸送から持ち込んでいるらしい。要確認?
}、これらは時間発展やObservable currentを表現するのに便利である。
実際、以下の関係式を示すことができる。
\begin{itembox}[l]{\textbf{Thm.}}
    時間発展やObservable currentはDynamical currentを用いて以下のように表される。
    \begin{align}
        \dv{\rho}{t} &= \sum_{\mu \in \mathcal{L}} \left( [J_{\mu}(\rho), S_{\mu}^{\dagger}] - [J_{\mu}^{\dagger}(\rho), S_{\mu}] \right) \\
        Q_{\mu}(\rho) &= \Tr[ J_{\mu}^{\dagger}(\rho) S_{\mu} ] + \Tr[ J_{\mu}(\rho) S_{\mu}^{\dagger} ]
    \end{align}
\end{itembox}
\textbf{Prf.}\\
\begin{align}
[J_{\mu}(\rho), V_{\mu}^{\dagger}]
&= i \zeta_{\mu} 
\left[ \Pi_{\mu}^{+} \rho (1 - \Pi_{\mu}^{-}) + \rho \Pi_{\mu}^{-} \right] V_{\mu}^{\dagger}
+ i V_{\mu}^{\dagger} \left[ \Pi_{\mu}^{+} \rho + (1 - \Pi_{\mu}^{+}) \rho V_{\mu}^{-} \right] \\
&= i \zeta_{\mu} (\rho V_{\mu}^{\dagger} - V_{\mu}^{\dagger} \rho)\\
&= i \zeta_{\mu} [\rho, V_{\mu}^{\dagger}] 
1\end{align}
を用いることで、
\begin{align}
[J_{\mu}(\rho), V_{\mu}^{\dagger}] - [J_{\mu}^{\dagger}(\rho), V_{\mu}]
&= -i [\zeta_{\mu} V_{\mu}^{\dagger} + \zeta_{\mu}^{*} V_{\mu}, \rho]
\end{align}
が得られる。また、
\begin{align}
[J_{\mu}(\rho), L_{\mu}^{\dagger}]
&= \frac{1}{2} \gamma_{\mu}^{+} (L_{\mu} \rho L_{\mu}^{\dagger} - L_{\mu}^{\dagger} L_{\mu} \rho)
+ \frac{1}{2} \gamma_{\mu}^{-} (L_{\mu}^{\dagger} \rho L_{\mu} - \rho L_{\mu} L_{\mu}^{\dagger}) 
\end{align}
を用いることで、
\begin{align}
[J_{\mu}(\rho), L_{\mu}^{\dagger}] - [J_{\mu}^{\dagger}(\rho), L_{\mu}]
&= \gamma_{\mu}^{+} 
\left( L_{\mu} \rho L_{\mu}^{\dagger} - \frac{1}{2} \{ \rho, L_{\mu}^{\dagger} L_{\mu} \} \right)
+ \gamma_{\mu}^{-}
\left( L_{\mu}^{\dagger} \rho L_{\mu} - \frac{1}{2} \{ \rho, L_{\mu} L_{\mu}^{\dagger} \} \right)
\end{align}
が得られる。これらを合わせることで、時間発展の表式が得られる。\qed\\

\section{内積構造}
上で導いた時間発展の表式は双対構造を誘導する。
dynamical currentの空間とエルミート演算子の空間のそれぞれで内積を導入する。
dynamical currentの空間$\mathfrak{J}:= \text{opr}(H)^{\mathcal{L}} \ni \mathbb{A},\mathbb{B}$上の内積を
\begin{align}
    \ev{\mathbb{A} , \mathbb{B}} := \sum_{\mu \in \mathcal{L}} \qty(\Tr[ A_{\mu}^{\dagger} B_{\mu}] + \Tr[A_{\mu} B_{\mu}^{\dagger}]) \in \mathbb{R}
\end{align} 
で定義する。
\footnote{
    実数になってくれるのは$\Tr[X] = \Tr[X]^*$と$z+z^* \in \mathbb{R}$が成り立つため。
}
また、エルミート演算子の空間$\text{Herm(H)} \ni A,B$上の内積を
\begin{align}
    \ev{A, B} := \Tr[A B] \in \mathbb{R}
\end{align}
で定義する。

ここで、グラディエント$\nabla_{\mathbb{S}} : \text{Herm}(H) \to \mathbb{J}$
を、
\begin{align}
    \nabla_{\mathbb{S}} A &= \{[A, S_{\mu}]\}_{\mu \in \mathcal{L}}
\end{align}
と定義する。
また、ダイバージェンス$\nabla^*_{\mathbb{S}} : \mathbb{J} \to \text{Herm}(H)$
を、
\begin{align}
    \nabla^*_{\mathbb{S}} \mathbb{A}&=  \sum_{\mu \in \mathcal{L}} \qty([A_{\mu}, S_{\mu}^{\dagger}] - [A_{\mu}^{\dagger}, S_{\mu}])
\end{align}
と定義する。
このとき、これらは双対であることが示される。\\
$(\because)$\\
\begin{align}
    \ev{\nabla_{\mathbb{S}} A , \mathbb{B}} &= \sum_{\mu \in \mathcal{L}} \qty( \Tr[ [A, S_{\mu}]^{\dagger} B_{\mu}] + \Tr[ [A, S_{\mu}] B_{\mu}^{\dagger}]) \\
    &= \sum_{\mu \in \mathcal{L}} \qty( \Tr[ S_{\mu}^{\dagger} A- AS_{\mu}^{\dagger}] B_{\mu} + \Tr[AS_{\mu} - S_{\mu} A] B_{\mu}^{\dagger}) \\
    &= \sum_{\mu \in \mathcal{L}} \qty( \Tr[ A (B_{\mu} S_{\mu}^{\dagger} - S_{\mu}^{\dagger} B_{\mu})] + \Tr[ A (S_{\mu} B_{\mu}^{\dagger} - B_{\mu}^{\dagger} S_{\mu})]) \\
    &= \Tr[ A \sum_{\mu \in \mathcal{L}} \qty( [B_{\mu}, S_{\mu}^{\dagger}] - [B_{\mu}^{\dagger}, S_{\mu}])] \\
    &= \ev{A, \nabla^*_{\mathbb{S}} \mathbb{B}}
\end{align}
\qed\\

このもとで、時間発展は
\begin{align}
    \dv{\rho}{t} &= \nabla^*_{\mathbb{S}} \mathbb{J}(\rho)
\end{align}
と表される。

\section{詳細つり合い条件の下でのEPR}
以下、局所詳細つりあい条件
\begin{align}
    k_{\text{B}} \ln \frac{\gamma_{\mu}^{+}}{\gamma_{\mu}^{-}} = \sigma_{\mu}
\end{align}
を仮定する。
このとき、着目系/環境系/全体のEPRを以下のように定義する:
\begin{align}
    \dot{\sigma}_{\text{sys}} &= -k_{\text{B}} \Tr[ \dot{\rho}(t) \ln \rho(t)] \\
    \dot{\sigma}_{\text{bath}} &= \sum_{\mu \in \mathcal{L}_{L}} \sigma_{\mu} Q_{\mu}(\rho(t)) \\
    \dot{\sigma}_{\text{tot}} &= \dot{\sigma}_{\text{sys}} + \dot{\sigma}_{\text{env}}
\end{align}
また、Thermodynamic force op.を
\begin{align}
    F_{\mu}(\rho) 
    \begin{cases}
        0 & (\mu \in \mathcal{L}_{H}) \\
        \sigma_{\mu} L_{\mu} + k_{\text{B}}[L_{\mu}, \ln \rho] & (\mu \in \mathcal{L}_{L})
    \end{cases}
\end{align}
と定義する。
このとき、
\begin{align}
    \dot{\sigma}_{\text{tot}} &= \ev{ \mathbb{J}(\rho) , \mathbb{F}(\rho) }
\end{align}
が成り立つ。\\
$(\because)$\\
\begin{align}
    \dot{\sigma}_{\text{sys}} &= -k_{\text{B}} \Tr[ \dot{\rho}(t) \ln \rho(t)] \\
    &= -k_{\text{B}} \Tr[ \nabla^*_{\mathbb{S}} \mathbb{J}(\rho) \ln \rho(t)] \\
    &= -k_{\text{B}} \ev{ \nabla^*_{\mathbb{S}}  \mathbb{J}(\rho) , \ln \rho(t)} \\
    &= -k_{\text{B}} \ev{  \mathbb{J}(\rho) , \nabla_{\mathbb{S}} \ln \rho(t)} \\
    &= -k_{\text{B}} \sum_{\mu \in \mathcal{L}} \qty( \Tr[ J_{\mu}^{\dagger}(\rho) [\ln \rho, S_{\mu}] ] + \Tr[ J_{\mu}(\rho) [\ln \rho, S_{\mu}^{\dagger}] ]) \\
    &= k_{\text{B}} \sum_{\mu \in \mathcal{L}_{L}} \qty( \Tr[ J_{\mu}^{\dagger}[L_{\mu}, \ln \rho] ] + \Tr[ J_{\mu}(\rho) [\ln \rho, L_{\mu}^{\dagger}] ]) 
\end{align}
が成り立つ。
\footnote{ただし、
\begin{align}
    [J_{\mu}, V_{\mu}^{\dagger}] &\propto [\rho, V_{\mu}^{\dagger}]\\
    [J_{\mu}, V_{\mu}] &\propto [\rho, V_{\mu}]\\
    \Tr[\ln\rho [\rho, V_{\mu}^{\dagger}]] &= 0 \\
    \Tr[\ln\rho [\rho, V_{\mu}]] &= 0
\end{align}
が成り立つため、Hamiltonian項は寄与しない。
}

また、
\begin{align}
    \dot{\sigma}_{\text{bath}} &= \sum_{\mu \in \mathcal{L}_{L}} \sigma_{\mu} Q_{\mu}(\rho(t)) \\
    &= \sum_{\mu \in \mathcal{L}_{L}} \sigma_{\mu} \qty( \Tr[ J_{\mu}^{\dagger}(\rho) L_{\mu} ] + \Tr[ J_{\mu}(\rho) L_{\mu}^{\dagger} ]) 
\end{align}
を合わせることで、
\begin{align}
    \dot{\sigma}_{\text{tot}} &= \sum_{\mu \in \mathcal{L}_{L}} \qty( \Tr[ J_{\mu}^{\dagger}(\rho) (\sigma_{\mu} L_{\mu} + k_{\text{B}} [L_{\mu}, \ln \rho] ) ] + \Tr[ J_{\mu}(\rho) (\sigma_{\mu} L_{\mu}^{\dagger} + k_{\text{B}} [L_{\mu}^{\dagger}, \ln \rho] ) ]) \\
    &= \ev{ \mathbb{J}(\rho) , \mathbb{F}(\rho) }
\end{align}
が成り立つ。
\qed\\

2つの超演算子$\mathbb{B}^{+}, \mathbb{B}^{-} $について、
\begin{align}
    (\mathcal{M}_{\mathbb{B}^{\pm}} )_{\mu} (\mathbb{A}) := \int_{0}^{1} \dd{s} (B_{\mu}^{-})^{s} A_{\mu} (B_{\mu}^{+})^{1-s}
\end{align}
を定義する。
ここで、
\begin{align}
    B_{\mu}^{\pm} = \sum_{n} b_{n}^{\mu\pm} \ketbra{\chi_{n}^{\mu\pm}}
\end{align}
と展開すると、
\begin{align}
[\mathcal{M}_{\mathbb{B}^{\pm}}(\mathbb{A})]_{\mu}
&=
\sum_{n m}
\frac{b_{n}^{\mu -} - b_{m}^{\mu +}}
{\ln (b_{n}^{\mu -} / b_{m}^{\mu +})}
\mel{\chi_{n}^{\mu -}}{A_{\mu}}{\chi_{m}^{\mu +}}
\ket{\chi_{n}^{\mu -}}\bra{\chi_{m}^{\mu +}}
\end{align}
と計算できる。
このとき、以下の関係が成り立つ。
\begin{align}
    \mathbb{J}(\rho) &= \mathcal{M}_{\mathbf{\Gamma}^{\pm}}(\mathbb{F}(\rho)\\
    \mathbf{\Gamma}_{\mu}^{\pm} &=\{\Gamma_{\mu}^{\pm}\}_{\mu \in \mathcal{L}} \\
    \Gamma_{\mu}^{\pm} &= \frac{1}{2}\gamma_{\mu}^{\pm} \rho
\end{align}
\textbf{Prf.}\\
詳細つり合い条件を用いることで、
\begin{align}
F_{\mu}(\rho)
&= \sigma_{\mu} L_{\mu} + k_{\mathrm{B}} [L_{\mu}, \ln \rho] \\
&= k_{\mathrm{B}} \left[ L_{\mu} \ln(\gamma_{\mu}^{+} \rho) - \ln(\gamma_{\mu}^{-} \rho) L_{\mu} \right] \\
&= k_{\mathrm{B}} \left[ L_{\mu} (\ln \Gamma_{\mu}^{+}) - (\ln \Gamma_{\mu}^{-}) L_{\mu} \right]
\end{align}
となる。これを用いて、
\begin{align}
\int_{0}^{1} (\Gamma_{\mu}^{-})^{s}
\left[ L_{\mu} (\ln \Gamma_{\mu}^{+}) - (\ln \Gamma_{\mu}^{-}) L_{\mu} \right]
(\Gamma_{\mu}^{+})^{1-s} \, ds
&= -\int_{0}^{1} \frac{d}{ds}
\left( e^{s \ln \Gamma_{\mu}^{-}} L_{\mu} e^{(1-s) \ln \Gamma_{\mu}^{+}} \right) ds \\
&= L_{\mu} \Gamma_{\mu}^{+} - \Gamma_{\mu}^{-} L_{\mu} \\
&= \frac{1}{2}\gamma_{\mu}^{+} L_{\mu} \rho - \frac{1}{2}\gamma_{\mu}^{-} \rho L_{\mu}
= J_{\mu}(\rho)
\end{align}
が成り立つ。\qed\\

このとき、前の話から
\begin{align}
\dot{\Sigma}_{\mathrm{tot}}(\rho)
&= \ev{\mathbb{F}(\rho), \mathbb{J}(\rho)}\\
&= \ev{\mathbb{F}(\rho), \mathcal{M}_{\Gamma^{\pm}}(\mathbb{F}(\rho))}\\
&= \ev{\mathbb{J}(\rho), \mathcal{M}_{\Gamma^{\pm}}^{-1}(\mathbb{J}(\rho))}
\end{align}
である。
ここで、内積
\begin{align}
\ev{\mathbb{A}, \mathbb{C}}_{\mathbb{B}^{\pm}}
&:= \ev{\mathbb{A}, \mathcal{M}_{\mathbb{B}}(\mathbb{C})} \in \mathbb{R},
\end{align}
を以下のように導入すると、
\begin{align}
\ev{\mathbb{A}, \mathbb{A}}_{\mathbb{B}^{\pm}} &\ge 0,
\end{align}
をきちんと満たしてくれる。したがって、
\begin{align}
\dot{\Sigma}_{\mathrm{tot}}(\rho)
= \ev{\mathbb{F}(\rho), \mathbb{F}(\rho)}_{\Gamma^{+}}
\ge 0
\end{align}
が成り立つ。

\end{document}